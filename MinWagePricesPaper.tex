\documentclass[11pt]{article}

\usepackage{blindtext}
\usepackage[pdftex]{graphicx}
\usepackage{fancyhdr}
\usepackage[margin=.8in]{geometry}
\usepackage[hang]{footmisc}
\usepackage{lipsum}
 \usepackage[flushleft]{threeparttable}
 \usepackage{tabularx} 
\usepackage{apacite}
\usepackage{float}
\usepackage{caption}
\usepackage{subcaption}
\usepackage{amsmath}
%\pagestyle{headings}
%\fancyhf{}
%\rhead{djfklas}
%\lhead{dkjfalj}
%\rfoot{ Page \thepage}
\usepackage{breqn}
\usepackage{titlesec}
\usepackage[labelfont={bf}]{caption}
%\usepackage{hyperref}
\usepackage{setspace}
\interfootnotelinepenalty=10000
\usepackage{arydshln}
\usepackage{rotating}
\usepackage{amsfonts}
\usepackage{bbm}
\usepackage{changepage}
\usepackage{pdflscape}
\doublespacing
\newcolumntype{Y}{>{\centering\arraybackslash}X}

\def\sym#1{\ifmmode^{#1}\else\(^{#1}\)\fi}

\rfoot{ Page \thepage}
%
%\titleformat{\section}
%  {}{\thesection}{1em}{\textbf}
%  
% \titleformat{\subsection}
%  {\scshape}{\thesubsection}{1em}{\textbf}
%  
%  
%  \titleformat{\subsubsection}
%  {\scshape}{\thesubsubsection}{1em}{\textbf}
  
\newcommand\blfootnote[1]{%
  \begingroup
  \renewcommand\thefootnote{}\footnote{#1}%
  \addtocounter{footnote}{-1}%
  \endgroup
}

\title{A New Angle on the Effects of a  Minimum Wage Increase On Restaurants:  Price Pass Through, Quality Changes, and Border Effects}
\author{Chelsea Crain \footnote{Department of Economics, University of Iowa, Iowa City, IA, \textit{Email: }chelsea-crain@uiowa.edu} }

\begin{document}

\maketitle

\begin{abstract}
The stagnant federal minimum wage in the U.S. has spurred an unprecedented number of state, county, and citywide minimum wage laws. The high number and varying magnitudes of minimum wage increases combined with the growing online presence of restaurants across the country has provided a unique setting in which to analyze the effects of a minimum wage increase on restaurants' prices and quality. Using a novel dataset comprised of menu item and restaurant quality information from thousands of east coast establishments across three states, I estimate the effects of varying levels of minimum wage increases enacted at the start of 2017. I find that prices rise 0.3\% - 1.2\% in response to a 10\% increase in the minimum wage. These price pass through effects are consistent across restaurant characteristics but are heterogeneous across item type. Further, the magnitude of price pass through is significantly lower for restaurants near the border of a minimum wage policy region, suggesting that local minimum wage policies negatively effect businesses on the borders. Finally, I find that customer perceived quality of restaurants is positively related to increases in the minimum wage for restaurants with high rating levels prior to a minimum wage hike, but negatively related for restaurants with low rating levels prior to a minimum wage hike. 


\end{abstract}

\newpage

\section{Introduction}
The minimum wage in the United States has remained stagnant at \$7.25 for almost a decade, which is over 30\% lower in real terms than the federal minimum wage in 1970. This stagnation of federal minimum wages for low-skilled workers has led to a nationwide movement to increase the minimum wage at a local level, with policies varying at the state, county, and city level. Minimum wage changes at the local level continue to be at the forefront of political debate across the country, as there is no clear consensus in the minimum wage literature about the total effects of a local minimum wage hike on businesses.  

A great deal of research has been done on the effects of minimum wage increases on prices, but the unique nature of the dataset and policies that I use in this paper allow for more extensive analysis of minimum wage related questions. These additional areas of study include heterogeneity in pass through across restaurant characteristics and item type, changes in restaurant quality, and the existence of border effects. The restaurant industry is the primary focus of this minimum wage study due to the fact that nearly three-fifths of all workers paid at or below the minimum wage are employed in the service industry \cite{charsmw}. I examine restaurants in three contiguous states on the East coast that all increased their respective minimum wages (though by differing magnitudes) on the first day of 2017. I examine the impact of the minimum wage changes using restaurant menus from Yelp.com and Grubhub.com. Using a longer time period and a larger sample with Yelp.com data, I estimate the impact on prices, customer-perceived quality, and border effects. One potential downside of the Yelp.com data is that restaurant owners may not keep the online menu prices current. To examine this potential concern, I utilize data from Grubhub.com, a website through which consumers directly order, therefore creating an incentive for restaurant owners to keep prices up to date.

Using these data I estimate the overall price pass through at the restaurant level due to a 10\% increase in minimum wage to be between 0.3\% and 1.2\%, depending on the subsample analyzed. These estimates are consistent with previous findings in the literature. The magnitude of the price pass through estimate is heterogeneous across restaurant type, with low employee and low quality restaurants showing significantly higher pass through. The magnitude of the price pass through also varies at the item level, where items in categories such as ``Popular'' and ``Sandwich'' show a pass through level of over 0.6\%, but items in categories such as ``Soup/Salad'' and ``Dessert'' show significantly lower levels of pass through, with estimates below 0.4\%. This is an important indication for studies in the literature that only examine a small number of items on a menu over time. 

Using the consumer rated Yelp stars values, I estimate the impact of a 10\% minimum wage increase on the customer rating of restaurants. I find a bimodal effect of an increase in the minimum wage on restaurants. Restaurants that were rated at the mean (3.5) or below prior to the minimum wage increase saw a significant decrease in the star rating given to them by consumers, where restaurants that started at ratings above 3.5 stars saw a positive effect on their consumer ratings due to the increase in minimum wage. The Yelp star rating of a restaurant has been shown to significantly impact a restaurant's profit, thus these estimates are economically significant.  

Lastly, I examine the extent to which a restaurant's proximity to a minimum wage border affects the level of price pass through. These border effects are an area of concern for policy makers when it comes to enacting local minimum wage laws. For restaurants located within twelve minutes of a bordering area that is facing a lower minimum wage hike, I find a significant relationship between the restaurant's proximity to the border and the level of price pass through. Specifically, I estimate that a restaurant ten minutes further away from the border will increases prices by 0.2 percentage points more than a restaurant on the border. These border effects suggest that a local minimum wage increase impedes the ability of restaurants to fully pass through prices to consumers due to the spacial proximity to competitors who are not facing an increase in minimum wage. 


\section{Motivation}
The nationwide movement to increase the stagnant minimum wage for low skilled workers, commonly denoted ``The Fight for \$15'' \cite{ff15}, led to an unprecedented number of local minimum wage laws. A local law, for this study, is defined as one mandated at the county or city level. In 2012 there were only five local minimum wage laws across the country, but by the beginning of 2017 there were over 40 \cite{localmws}. These local minimum wage laws, however, have come with a considerable amount of political debate. For example, two local areas in Iowa enacted minimum wage laws at the start of 2017 but these policies were retracted after the state governor signed a law barring any local minimum wage ordinances \cite{ia}. The increase and variation in minimum wage policies has opened the door for further research to be done on the effects of such policies. 

The main area of interest in this paper is the extent to which a minimum wage increase impacts restaurant prices. The price of restaurant food has become an integral part of the American budget. In 2015, for the first time in history, Americans spent more money eating out at restaurants than they did on groceries \cite{usda}. And although low skilled workers receive a higher income after minimum wage hikes, low income workers purchase more minimum wage produced goods \cite{macurdy2000increasing}. This suggests that changes in restaurant prices affect the relative purchasing power of low income workers more than the average customer. 

Previous literature has examined the question of price pass through extensively, beginning with the publication of Card and Kruger's well known study about the effects of minimum wage changes on the fast food industry in 1994\nocite{card1994minimum}. Card and Kruger found a small positive effect on employment and no effect on prices after an increase in the minimum wage. These findings contradicted the textbook model of competitive labor markets, which predicts an increase in output prices and a decrease in employment. In response to this paper, many studies were published analyzing the existence of monopsony power in the labor market \cite{manning1995we, rebitzer1995consequences, burdett1998wage, bhaskar1999minimum}, as a monopsony model predicts an increase in employment and a decrease in prices. Neumark and Wascher (2006)\nocite{neumark2006minimum} in their survey of the literature conclude that the most rigorous and reliable studies have found significant price increases but small employment decreases. As Aaronson, French and MacDonald (2008) conclude\nocite{aaronson2008minimum}, the presence of significant increases in price in response to minimum wage increases is evidence against the prevalence of monopsony power in the labor market. It is therefore expected that an increase in minimum wage will lead to an increase in output prices.
% They furthermore argue that as long as factor markets are competitive, monopolistically competitive product markets also predict a decrease in employment and an increase in output prices. 

The magnitude of the price pass through to consumers after a 10\% increase in minimum wage varies in the literature between 0.4\% and 1.5\%. Allegrotto and Reich (2015) use full online menus to analyze prices before and after a local minimum wage increase in San Jose and find pass through to be 0.58\%. Aaronsen, French and MacDonald (2008) utilize the CPI food away from home data which is comprised of several bundles of food, usually the equivalent of a meal, at a variety of establishments across the country to examine price pass through of state and federal minimum wage changes. The authors estimate a price pass through of 0.7\%. Basker and Khan (2013) estimate price pass through at 0.9\% for McDonald's Quarter Pounders and a Pizza Hut regular cheese pizza due to state level variation in the minimum wage. These estimates vary with the types of items and time period analyzed but show that restaurants consistently pass through the increased labor costs in the form of higher output prices. 





%Although there is evidence against the prevalence of monopsony power in the labor market, there is still a discrepancy between what is predicted in the models and what is seen in the data.  Under labor market models of competition, monopsonistic competition, and even efficiency wages, a relatively large increase in price should be accompanied by a relatively large decrease in employment \cite{aaronson2008minimum}. Thus the question begs, in what other ways are restaurants offsetting the increases in labor costs? Some suggestions that have been made in the literature include changes in non-wage compensation and the ``hungry teenager hypothesis''\cite{kennan1995elusive}. This hypothesis suggests that the increase in income seen by minimum wage workers causes them to purchase more minimum wage products, thus offsetting some of the dis-employment effects. However, MacCurdy and O'Brien-Strain (2000) estimate that this could account for only 10 to 30 percent of the employment loss\nocite{macurdy2000increasing}. 

One way that restaurants could absorb the increase in labor costs, other than by increasing price or decreasing employment, is to decrease quality. For example, changes in overall restaurant quality after a minimum wage hike could come in the form of better service due to efficiency wages, reductions in portion size, or decreases in food quality.  Unlike data used in previous studies, the unique dataset used in this paper allows for analysis of any changes in customer-perceived quality due to a change in minimum wage using the star rating of restaurants on the review platform, Yelp. 

%Two other ways that restaurants may cut corners to make up for increased labor costs is to decrease the number of items offered on the menu, or to decrease hours of operation. Most previous studies have focused on a few menu items, or a few bundles of items and are therefore not able to analyze changes within the full menus. In addition to full menu analysis, the dataset used in the present study also provides the ability to look at the type of menu item, such as an appetizer versus a main dish, which is unique to the literature. 


From a policy perspective, one of the proposed negative effects of a minimum wage increase is the inability of restaurants close to the border of the city or state to account for these increased input costs in the form of higher prices without loosing business. It is thought that restaurants facing minimum wage increases close to a border where competitors on the opposite side of the border are not facing an increase in labor costs may have to keep prices artificially low in order to compete. These effects are referred to as border effects, and can be measured by the existence of a significant relationship between the distance a restaurant is to a bordering area with a different minimum wage increase and the magnitude of the price increase. The existence of border effects are crucial when it comes to policy evaluation and fully understanding how local businesses are affected by changes in minimum wage policy. 

%To think of this effect in more detail, imagine a simplified world in which there are two restaurants  on opposite sides of a state line competing in a market, where one state is increasing the minimum wage and the other is not. The business who is facing no increase in minimum wage knows about the increase in labor costs to their competitor, and thus as Chicu, Vickers, and Ziebarth (2012) find, the increase in the competitor's cost may have a positive effect on the restaurant's own price. This effect may also hold true in the opposite direction, where a restaurant facing a minimum wage increase that is close to a border where competitors are facing no increase may have a negative effect on the price increase. If there are border effects, these effects would be predicted to dissipate as the distance from the restaurant to the border increases. 

%The existence of border effects have different implications depending on the side of the border in which they exist. If there are significant border effects on the side of the border where there is a lower minimum wage increase, this would imply that workers and consumers are hurt by the minimum wage increase of the other state. Workers and consumers would be hurt since they are seeing an increase in prices but none of the employees are receiving a higher wage. If there are significant border effects on the side of the border where there is a higher minimum wage increase, this would imply that businesses are hurt by the minimum wage increase because they are forced to keep prices artificially low in order to compete with other businesses.

%Therefore, for the purposes of this paper it will be assumed that the labor market is competitive and that there is monopolistic competition in the output market following Dixit-Stiglitz (19977)\nocite{dixit1977monopolistic}.





\section{Minimum Wage Laws}
On January 1, 2017, three contiguous states on the east coast increased their minimum wage at differing magnitudes, with a variety of levels of increases within the state of New York. Table 1 reports the increases in minimum wage in these areas, which range from 0.72\% to 22\%\footnote{The last three columns of Table 1 report any changes in the tipped minimum wage. Although there was a change in the tipped minimum wage in Massachusetts, I focus only changes in the regular minimum wage for the analysis in this paper.}. These states provide a useful setting for minimum wage analysis as they are in the same geographic region and have similar economic, demographic, and political characteristics. Each area also faced the changes in minimum wage over the same time period. 

In April of 2016, New York (NY) became the second state, after California, to pass a law that would incrementally raise the minimum wage for all workers to \$15/hour.\footnote{In 2015, NY passed a minimum wage law only applicable to fast food restaurants, increasing the minimum wage each year for fast food workers \cite{nyff}. Do to the small sample size and unique nature of fast food restaurants in the data, I exclude all fast food restaurants from the analysis.} In this law, which applies to all non-fast food restaurants, the degree of the minimum wage increase is based on the type and location of the establishment \cite{nybill}. The first two rows of Table 1 report minimum wage changes for restaurants in New York City (NYC). Workers in NYC at establishments with more ten employees, denoted as large restaurants, saw a 22\% increase. Workers in NYC at small restaurants received an increase of 16.67\%. The third column of Table 1 includes restaurants of all sizes in the three contiguous counties outside of NYC: Nassau, Suffolk, and Westchester. These employees saw an increase of 11.11\%. The final NY group, NYC Upstate, includes restaurants of all sizes elsewhere in the state of NY who received a 7.78\% increase in minimum wage.

Two of the states contiguous to NY saw changes in their own state-wide minimum wage laws as of January 1, 2017. %In July of 2014, Connecticut (CT) passed a law that would increase the minimum wage for all workers in small increments each year until 2017, where it reached \$10.10 giving minimum wage workers a 5.21\% increase \cite{ctbill}.
In 2014, Massachusetts (MA) passed a bill that increased the minimum wage by \$1 a year starting at the beginning of 2015. The state saw the final increase relating to this bill on January 1 of 2017, which increased the minimum wage by 10.00\% \cite{mabill}. %Vermont (VT) passed a minimum wage law in 2013 that increased the minimum wage for the state each year from 2014-2018, providing workers with a 4.2\% increase in minimum wage at the beginning of 2017 \cite{vtbill}. 
In the spring of 2016, New Jersey (NJ) proposed a minimum wage law similar to that of NY that would put the state on track to a \$15 minimum wage. The bill passed through the house and the senate, but in September of 2016, NJ governor  Chris Christie vetoed the bill, stating ``...[this bill] fails to consider the capacity of businesses, especially small businesses, to absorb the substantially increased labor costs it will impose''\cite{njveto}. Thus in 2017 NJ increased the state minimum wage by only 0.72\%, a yearly adjustment for inflation \cite{njbill}. %In Pennsylvania (PA), workers saw no increase in the minimum wage at the beginning of 2017, as the state stayed at the federal level of \$7.25 7 \cite{panobill}. However, PA governor Tom Wolf called for an increase of the minimum wage to \$12/hour in his state budget proposal in February 2017 \cite{govwolf}. If passed, this would increase the PA minimum wage by an unprecedented 65\%. 
The state of NJ is therefore a strong counterfactual to NY and MA due to the similar political sentiment shown by the attempts at increasing the minimum wage. 




\section{Data}

Three primary datasets will be used in this analysis. The first, and most extensive,  is a panel dataset comprised of restaurant menu  and quality information from Yelp.com. The second is a supplementary panel dataset also comprised of restaurant menu information from Grubhub.com.  The third is a dataset providing detailed restaurant business information from ReferenceUSA. These datasets will be utilized to determine the magnitude of price pass through to consumers, how this pass-through varies by restaurant and item specific characteristics, and how the minimum wage affected customer perceived quality.

\subsection{Yelp}

Yelp.com is a website in which consumers can find restaurant information including customer reviews, hours of operation, price range, and full menus. Yelp was founded in 2004, and currently has an average of 72 million monthly visitors with over 115 million reviews written \cite{yelpstat}. The first wave of Yelp data collection was in April 2016, the second wave in July 2016, and the third wave in October 2016. Two waves of data collection occurred after the minimum wage increases, in January 2017 and April 2017.

To collect a geographically representative sample, a list comprised of areas of interest was compiled spanning all three states and focusing more heavily on larger cities. I wrote a webscraping program that iterated through each area of interest, collecting the home page and menu page for each restaurant in the area. The Yelp homepage of each restaurant contains information about the restaurant including address, phone number, Yelp star rating, price range, food category and hours of operation. Approximately one third of these restaurants provide a full menu on Yelp. Some restaurants provide an external link to a menu, but since these menus are not formatted uniformly, the menu information cannot be correctly parsed and thus for the sake of this dataset these restaurants fall into the same category of those restaurants who do not provide an online menu. In a similar study using online menus, Allegrotto (2016) also found that on average about one third of restaurants posted full online menus\nocite{allegretto2016local}. These externally formatted menus were hand entered for one round of the scrape, and characteristics of these restaurants were not statistically different than the Yelp formatted menus. Since each Yelp menu is uniformly written, I created a second program that parsed the item and price information into a usable dataset along with other restaurant characteristics. These menus provide item category and price information that is parsed into a usable dataset along with the other restaurant characteristics. Figure 1 shows the geographical distribution of the restaurants in the sample.

One concern with using Yelp data is the reliability and frequency with which the information for a restaurant is updated, as Yelp menus may be updated at the restaurant owner or manager's discretion. As a way to deal with this reliability issue, I take advantage of a food delivery service that is powered by Yelp, called Eat24. This service provides customers the ability to order food directly from the restaurant's Yelp menu page for delivery or pickup. The use of this service ensures that the menu items and prices are up to date since customers are ordering directly from the provided menu information. Each restaurant in the dataset has an indicator for whether or not the business utilizes the Eat24 food delivery service. 

Yelp users provide online reviews of restaurants and assign them an overall star rating on a scale of 1 to 5, with 1 being extremely poor and 5 being outstanding. Although the number of stars that a restaurant has earned is determined by self-selected reviewers, the number of Yelp stars has been found to be a reliable predictor of actual quality as well as an important determinant of profit for restaurants. In a study comparing Yelp star ratings of hospitals to an industry standard assessment of quality, Bardach et. al (2014) found that Yelp stars were significantly related to patient care and health outcomes. Luca (2011) used a regression-discontinuity design to analyze the impact of a change in the Yelp star rating on restaurants, and found that a one-star increase in the Yelp rating led to a 5-9\% increase in revenue. Another concern of using Yelp stars as a point of analysis regards restaurants writing fake reviews-- good reviews of themselves and bad reviews of competitors. Yelp uses a proprietary algorithm in an attempt to filter out fake reviews which are in turn not included in the Yelp star rating. Although some fake reviews may still exist, I will assume for the analysis that the any fake reviews on quality are not correlated with changes in the minimum wage. 

The rounded star rating that customers see prominently displayed on each restaurant's hoempage is the monthly average of all Yelp reviews rounded to the nearest half star.\footnote{If a restaurant has less than 10 reviews within a month, then the most recent reviews are added until the sample size reaches 10.} There is significant variation in the Yelp star rating for restaurants over the time period of the dataset, an indication that Yelp users are active in reporting the current quality of the establishments. Over 50 percent of restaurants see a change in star rating between any given observation period, and the average change given an increase (decrease) is 0.62 (0.61) stars.


\subsection{GrubHub} 

Founded in 2004, Grubhub.com is the largest online food ordering company in the U.S., providing its 7.7 million customers in over 1,100 cities accesss to delivery at over 45,000 locations \cite{grubstat}. This data was collected with the primary intention of validating the primary Yelp data. I collected menu information for all GrubHub restaurants in the areas of interest using a second webscrape in December 2016, January 2017, February 2017, March 2017 and April 2017. Although there are fewer restaurants on Grubhub than there are on Yelp, the GrubHub menu prices, which are necessarily up to date given the nature of the delivery service, provide an idea to what extent the Yelp data is generalizable. All restaurants on Grubhub provide a uniform menu and are therefore included in the dataset of parsed menu information. Grubhub also allows customers to rate the restaurants, but there are significantly fewer reviews given on Grubhub than there are on Yelp. Ratings on Grubhub may also reflect the delivery service and not the restaurant itself. Grubhub restaurants are primarily in the more highly populated areas and therefore sample sizes for the smaller cities are much lower in the Grubhub data.


\subsection{ReferenceUSA}

ReferenceUSA (RUSA), an Infogroup company that provides business data, is used to define more detailed characteristics of the restaurants. I collect the data for all businesses in the areas of interest that are categorized under the North American Industry Classification System (NAICS) as a restaurant. The restaurant level variables obtained from this dataset include sales volume, number of employees, limited service status, and franchise status. Since these data are updated on a yearly basis, the variables are used only as baseline characteristics of the establishments. 



%As a measure of the validity of the decision to use ReferenceUSA as the population I look to the New York City Department of Health and Mental Hygiene. The Department of Health and Mental Hygiene is required to keep a list of all business that perform any kind of food service, of which there are 26,047 listed. This list, however, contains businesses such as country clubs, banks, and concession stands, which would not be in the ReferenceUSA dataset by construction. In the ReferenceUSA dataset there are 24,228 restaurants. Therefore, although not a perfect match, I assume for the rest of the analysis that the ReferenceUSA dataset is representative of the true population.



\subsection{Data Construction and  Definitions}
%
%The restaurant characteristic definitions are key to understanding the effects of the minimum wage increases. A limited service (LS) restaurant is one in which customers generally order and pay for items before eating. A full service (FS) restaurant is defined as an establishment where customers sit down to eat, and where the food is served directly to the customers' table, and the customers pays after eating. The ReferenceUSA data contains the NAICS codes from which I can determine the LS or FS status of a restaurant. For this study, a chain restaurant is defined as having a franchise status in the ReferenceUSA dataset. A fast food (FF) establishment is defined as a restaurant that is both LS and a chain, the definition used in the New York minimum wage laws. A large restaurant will be defined as one employing more than 10 workers, which once again follows the definition used by the state of New York.
%
%Table 2 shows the sample construction of the dataset, where ReferenceUSA is used as the population. The proportion of LS and chain restaurants for both datatsets decreases at each level of sample construction. From this table it can be concluded that LS and chain restaurants are slightly less likely to be on Yelp, and significantly less likely to have a menu posted on Yelp that contains prices. This is consistent with Allegrotto and Reich (2015) who found that many chains do not post menus with store specific prices online.\nocite{allegretto2016local}\footnote{For example, McDonald's only posts menu prices on electronic menu boards inside each establishment.} There is also less motivation for chain restaurants to have complete and updated Yelp profiles, as they are not as strongly effected by the quality of their Yelp presence \cite{luca2011reviews}. From a political perspective, one of the biggest concerns of a minimum wage increase is the welfare of the small businesses. This dataset, although somewhat under-representative of the chain restaurants, has the ability to provide more insight into the effects of a minimum wage increase on these small businesses. %Table 3 reports sample sizes across minimum wage  groups.

To determine the minimum wage that the restaurants face, each restaurant sample is mapped to a county using the Census Geocoder. In NYC, the minimum wage that a restaurant faces is also dependent on the restaurant type. The variables in the RUSA dataset provide enough information to determine this but restrict the sample size due to limited matching between RUSA and the web scraped datasets. Only a little more than half of the total restaurants in Yelp and Grubhub are able to be matched with the RUSA dataset. This matching problem is consistent over multiple matching methods. Therefore, in this study I use the average minimum wage increase in NYC for all restaurants, regardless of size.\footnote{The results are similar when using the specific minimum wage increase by restaurant type in NYC.} Since there are significantly more restaurants that qualify as ``small" in NYC, using the un-weighted average of the minimum wage increase will only bias my estimates towards zero. 

For the Yelp and Grubhub menu data I create a balanced panel at the item level, only including restaurants and items that are in all waves of data collection. One concern with using a balanced panel is that firms could respond to changes in the minimum wage by changing the items offered. However, I find no significant relationship between changes in the total number of items offered at a restaurant and changes in minimum wage. Another concern is that firms may change the quality of the items in the balanced sample. These potential quality changes are addressed in Section 5. 

Summary statistics of the balanced panels aggregated at the restaurant level for Yelp and Grubhub are reported in Table 2 and Table 3, respectively. Limited service and franchise restaurants are a very small portion of the restaurants in these both data sets. Most previous studies in the literature have focused primarily on the fast food and chain restaurants, thus these data provide information on an under-represented sub population. It has also been shown in previous studies that limited service restaurants have a higher level of pass through because they employ more workers at a binding minimum wage. From these two tables it can be seen that restaurants update menus less frequently in the Yelp dataset, but the changes in price given an increase or decrease are larger. 

%An additional concern is the extent to which using only restaurants with Yelp formatted menus limits the generalizability of the analysis. The data from the menus that were not in the uniform Yelp format were hand entered for one wave of data collection. The mean price and number of items of restaurants for each group were not significantly different than the data from the Yelp formatted menus. 
%Each item is matches across time by name, making changes in an item name and removal of an item indistinguishable. I assume for analysis that any changes in name or removals of items are uncorrelated to changes in the minimum wage. 

The restaurants within in a minimum wage group are not necessarily the same restaurants across Yelp and Grubhub.  There are some restaurants that are in both datasets and these are discussed specifically in the analysis. Because of the small sample sizes and unique characteristics of the fast food restaurants in the dataset, all fast food restaurants will be excluded from the analysis presented in this paper. This removes 68 restaurants from the sample, 31 of which are from NYC. 

To analyze the existence of border effects, I first construct a distance matrix using Google API. For each restaurant, the distance matrix includes all restaurants on the opposite side of a border facing a different minimum wage increase that are within 12 miles. The driving distance in minutes is then calculated to each restaurant and the minimum is then recorded as the distance to a given border. This provides a finer measure of distance than raw miles. For the border effects analysis, all restaurants that are within 12 minutes are included, as Iacono et a. (2008) found that 90\% of Americans will only travel 12 minutes at maximum to go to a restaurant\nocite{drivingtime}.

%Table 4 provides summary statistics regarding price changes in Yelp menu items. ( These are all pre-min wage change. Need to add post-min wage change and discuss differences and similarities)




%To analyze border effects, multiple definitions in the data need to be made. Take any restaurant in the dataset for which there is price information available and which has been matched to the ReferenceUSA database. To find if there are any border effects in the positive direction, I first find all restaurants in the population that face a higher minimum wage increase than the sample restaurant. Out of those restaurants, I define a competitor to be a restaurant which has the same LS and chain characteristics of the sample restaurant. Finally, out of the group of restaurants in the population that are defined as competitors and who face a higher minimum wage than the sample restaurant, I use Google maps API to find the shortest driving time to a competitor.\footnote{I restrict the distance to be within 30 miles, as I am only interested in effects for establishments relatively close to a state or county border.} This driving time is the value that will be used as the ``distance to the nearest competitor facing a higher minimum wage increase.'' The same methods are used to find ``distance to the nearest competitor facing a lower minimum wage increase.''




\section{Price Pass-Through}

%\subsection{Price Pass-Through}

%\subsection{Model Predictions}
%Due to the evidence against the prevalence of monopsony power in the labor market \cite{manning1995we, rebitzer1995consequences, burdett1998wage, bhaskar1999minimum, aaronson2008minimum}, it will be assumed that factor markets are competitive and that product markets are competitive or monopolistically competitive. Both of these theoretical assumptions provide the same estimates for the purpose of calculating price pass through \cite{aaronson2008minimum}. If firms have a constant returns to scale production function, then, ceteris paribus, increases in labor costs will be proportionally passed on to consumers in the form of 
%\begin{center}
%\centering ($\% \Uparrow$ MW) $*$  \small($\frac{\mbox{MW Costs}}{\mbox{Labor Costs}}$)  $*$  ($\frac{\mbox{Labor Costs}}{\mbox{Total Costs}}$). 
%\end{center}
%\noindent The 2002 Economic Census for Accommodation and Foodservices  estimates the labor share of total costs, the third term, to be between 26 and 32 percent for restaurants in the U.S\nocite{census}. The second term, minimum wage's share of labor costs, is more difficult to estimate due to a lack of data. However, Aaronson and French (2007) use a detailed calculation method and estimate that minimum wage payments constitute approximately 17 to 33 percent of total wage payments in the industry\nocite{aaronson2007product}. Thus the lower bound estimate of price pass-through given a 10 percent increase in minimum wage is $10\% * 26\% * 17\% = 0.44\%$, and the upper bound estimate is $10\% * 33\% * 32\% = 1.06\%$. These estimates account for total price pass through which has been shown to occur up to four months after a minimum wage hike due to the sticky nature of prices. Therefore it is to be expected that the magnitude of the price pass through for the first month after the policy implementation will be slightly smaller than these estimates.


%Another way to estimate price pass through, which accounts for cost savings due to reduced turn over, is through earnings elasticity. Allegrotto, Dube, Reich and Zipperer (2015) estimate a minimum wage earnings elasticity of 0.28 and a turnover savings of 15 percent. Using an estimated labor cost share of total cost of 33 percent and multiplying that by the minimum wage earnings elasticity net of turnover savings, the price pass through is estimated at $0.59\%$ for a ten percent increase in minimum wage. Therefore it is to be expected that a 10 percent increase in the minimum wage would increase prices on average between about 0.5 and 1.0 percent. 

\subsection{Analytical Model}

The first question to answer in this setting is to what extent the increases in minimum wage are passed on to consumers through prices. The base model of price pass through at the restaurant level regresses the log change in price on the log change in the minimum wage,
\begin{dmath}
\Delta \ln p_{jkt} = \sum_{h=l}^{L}\beta_h \Delta \ln mw_{kt-h} + \gamma  P\_START_{j} + \zeta T\_BTWN_{jkt}   + \epsilon_k + \epsilon_m +\epsilon_{jkt}
\end{dmath}


\noindent where $p$ is the average price of items at restaurant $j$ in minimum wage group $k$ in observation wave $t$.\footnote{Each restaurant is location specific.} I allow for a flexible price response from restaurants by including contemporaneous and lagged changes in minimum wage. For the specifications using the Yelp data, one lead period is also included ($l=-1, L=1$) in order to analyze the existence of any relationship between price increase and minimum wage group before the policy implementation.\footnote{Including a second lead period does not change the results and estimates insignificant price pass through.} Although all groups knew of the policy changes by April 2016, it is unlikely that restaurants responded to the impending wage hikes more than four months in advance. For example, Aaronson et al. (2008) found that restaurants do not respond to changes in the minimum wage more than two months ahead of implementation. Therefore the estimates of this lead term are a good indication of the existence of policy endogeneity. For the Grubhub specification, no lead and three lag periods are included ($l=0, L=3$). No lead period can be included since the first period of observation is in December. Although there are three lags included in the Grubhub specification, these time periods encapsulate price changes over the same four month period as the one period lag in the Yelp specification. 

The average price at a restaurant before the change in minimum wage is the primary means in which to characterize restaurants in the Yelp dataset. Thus $P\_START_{j}$ is the average price at restaurant $j$ in the first observation period, April 2016. Since the data was collected using a web scrape, there was some variability in the timing of data collection for each restaurant in each wave. The variable $T\_BTWN_{jkt}$ is an integer representing the number of days between observations for a given restaurant, and $\epsilon_m$ is a fixed effect controlling for the month that the data was collected. Together, these two terms account for any differences in the timing of the data collection between waves as well as any seasonality. To test the extent to which restaurant characteristics, other than price, are a driving force of the price pass through, $\boldsymbol{X}_j$, a vector of RUSA variables including sales volume, employees, limited service status and franchise status, is included in some specifications. I assume that NJ is an appropriate counterfactual for NY and MA. As discussed in Section 2, NJ is geographically close and socioeconomically similar to both states that did increase the minimum wage. NJ also has a similar political sentiment as the state attempted to increase their own minimum wage at the start of 2017. I therefore assume that there are no other unobserved characteristics that are related to both the minimum wage increase and the increase in price. 




\subsection{Estimates}

The results of the statistical estimation model described above are reported in Table 4. All estimates are reported to be  interpreted as the percent change in price due to a 10\% increase in the minimum wage over the given time period. Observations are aggregated at the restaurant level, and errors are clustered at the minimum wage group level. The row titled Total Pass Through is a linear summation of the estimated coefficients in all relevant time periods. For the specifications using the Yelp data, the total pass through estimates are combinations of the October to January and the January to April estimates. Since the July to October estimate is only included in the model as a means of testing for policy endogeneity, this estimate is not included in the total pass through. It is also insignificant in all specifications. For the specifications using the Grubhub data, all time periods are included in the total pass through estimates. 

Column 1 reports estimates using the full Yelp sample of restaurants. There was significant pass through in both the contemporaneous and lagged time periods, for a combined pass through estimate of 0.33\%. Column 2 includes the vector of control variables from the RUSA dataset. These estimates are almost identical to the estimates reported with the full sample. Column 3 restricts the full Yelp sample to only those restaurants who changed the price of at least one item throughout the course of the dataset. The total pass through estimate for this subsample is significantly larger at 1.22\%. The fourth column is restricted to restaurants that employ the Eat24 food delivery service. This price pass through estimate, 0.60\%, is also significantly higher than the full sample. Column 5 reports the pass through estimates of restaurants that employ the Eat24 food delivery service and are present in the Grubhub sample. This price pass through estimate is insignificant and not statistically different from the full sample. 

The last three columns of Table 4 report estimates using the Grubhub dataset. The estimated pass through using the full Grubhbub dataset, as shown in column 6 is 0.91\%. Column 7 reports that adding the control vector of RUSA restaurant characteristics significantly increases the pass through estimates. The final column restricts the Grubhub sample to only restaurants that are also in the Yelp dataset and utilize the Eat24 food delivery service. The price pass through using this subsample is estimated at 0.60\%. It should be noted that due to the nature of the sample restrictions, columns 5 and 8 represent the same restaurants from two different data sources. 


To investigate heterogeneity of price pass through across restaurant characteristics, Table 5 reports the pass through estimates by highest and lowest quartiles of sales volume, employees, and star ratings. All columns use the Yelp data and the baseline specification. Once again, no significant pass through is reported from July to October, and the total pass through estimates are linear combinations of the relevant coefficients. Columns 1 and 2 compare restaurants by highest and lowest quartiles of sales volume. Although the high sales volume estimate is larger at 0.33\%, it is not significantly different from the pass through estimate of low sales restaurants, 0.19\%. Columns 3 and 4 compare restaurants by the number of employees. The pass through estimate of 1.01\% for low employee restaurants is significantly higher than the estimate of 0.32\% for high employee restaurants. Columns 5 and 6 compare restaurants by Yelp star rating in April 2016, before any minimum wage changes. The estimated price pass through for low star restaurants, 0.35\%, is significantly higher than the estimate for high star restaurants, 0.17\%.       

There may also be heterogeneity across item type, and given the nature of the dataset these differences can be explored. Table 6 reports pass through results at the item level using the Grubhub dataset.\footnote{Similar patterns exist in the Yelp dataset, but item categories are not as clear cut and so significantly more items fall into the ``other" category. Grubhub menus also do not provide a ``popular'' category.} The baseline specification of model (1) is used, except now an observation is an item. All Grubhub item are classified under an item category. Items in the ``popular" category are classified as such based on the frequency an item is ordered through Grubhub. Column 1 reports pass through estimates at the item level for all items, with a total pass through estimate of 0.48\%. The subsequent columns report pass through estimates for the seven most frequent item categories.\footnote{The categories not reported include; ``appetizers", ``pizza", ``kids", ``breakfast", and ``other".} Price pass through for ``popular" items, ``side" items, and ``sandwiches" was significantly higher than the average of all items, ranging frm 0.62-0.69\%. Price pass through  for ``soups", ``salads" and ``sandwiches'' were significantly lower than the average of all items, ranging from 0.33-0.37\%. Estimates for ``drinks'' and ``desserts'' were not significantly different than when using all food categories. 




\section{Quality}

\subsection{Model}
I now turn turn to the change in quality as an outcome variable. Figure 2 depicts the relationship between minimum wage increase and the change in star rating by the star rating of the restaurant in April 2016, prior to any minimum wage changes. As can be seen from the figure, there appears to be a negative relationship between minimum wage and customer perceived quality for those restaurants who started at a lower rating and a positive relationship for the higher rated restaurants. To investigate this relationship further, the following model is used to analyze the effect of a minimum wage increase on the Yelp star rating of a restaurant $j$ in minimum wage group $k$ at observation wave $t$:
\begin{dmath}
\Delta \ln(stars_{jkt}) = \alpha + \sum_{h=l}^{L}\beta_h \Delta \ln mw_{kt-h}  + \gamma  P\_START_{j} + \epsilon_k + \epsilon_t + \epsilon_{ijkt}
\end{dmath}
The variable $stars_{jkt}$ is the rounded half star rating of each establishment. Just as in the previous model, contemporaneous and lag terms for changes in the minimum wage are included to allow for a flexible response. A lead term is also included ($l=-1,L=1$) since the Yelp datat is used, as a check for any pre-trends between minimum wage group and changes in quality. The term $P\_START_j$, the average price at an individual restaurant in April 2016, is included as a control. The average price of items at a restaurant may provide the customer with an expectation of what the level of quality should be. These ex ante ideas of quality level could have an impact on changes in customer perceived quality. Fixed effects for observation time period and minimum wage group are included in all specifications.\footnote{Using the observation period fixed effects are the same as using month fixed effects in this model since the star ratings for each restaurant are collected in the same month for all restaurants.} I assume that there are no other unobserved characteristics of restaurants that influence both the minimum wage group that a restaurant belongs to and the customer perceived level of quality. This model estimated the full relationship between minimum wage increases and customer perceived quality. However, one may be interested to know the extent to which these quality changes are driven by customers' dissatisfaction of any increases in price. The change in price at each time period, $\Delta\ln(p_{jkt})$, is added to the model in some specifications to control for any changes in quality that are driven by increases in price. 

\subsection{Estimates}
The regression results of this model are presented in Table 7. All estimates are interpreted as the percent change in star rating due to a 10\% increase in minimum wage. The total percent change in star rating is a linear summation of the estimated coefficients from October `16 to January `17 and January `17 to April `17. Once again, the estimate from July to October is not included in the total change since it is only included in the model as a test of pre-trend relationships. The first two columns of the table report estimates using the full sample of Yelp restaurants. Although the estimates are negative, all are imprecisely measured. There is no difference in estimates when the change in price term is included. Columns 3-4 report the estimated change in star rating due to a minimum wage increase for restaurants that had a 2.5 rating in April `16.\footnote{Restaurants below a 2.5 rating and above a 4.5 rating are not analyzed as subsamples given that they are close to the lower and upper bounds, respectively, and so only have one direction to move.} The estimated relationship is more negative than the full sample, but once again imprecisely estimated. The estimates do not change once change in price are controlled for. The next two columns, 5-6, report the estimated relationship for restaurants that started at a 3.0 star rating. For these restaurants, a 10\% increase in minimum wage is associated with a 3.1\% decrease in star rating, which is consistent when change in price is included. Columns 7-8 report estimates for restaurants that began with a 3.5 star rating, the average in the sample. For the specification both with and without controlling for price changes, the estimated relationship is -1.1\%. As seen in columns 9-12, the estimated relationship for restaurants that started at a 4.0 or 4.5 star rating saw a positive relationship with regards to changes in the minimum wage. Restaurants starting at a 4.0 rating saw a 2.4\% increase in star rating due to a 10\% increase in the minimum wage, and restaurants starting at a 4.5 rating aw a 1.5\% increase in star rating. Consistent with the previous findings, controlling for price changes did not significantly change either estimate. 

%
%To gain further insight into where changes may be happening throughout the distribution of restaurants, I turn to an ordered probit. Based on the exact Yelp rating in a month, I define the outcome $s$ as 
%\begin{equation}
%   s=\left\{
%                \begin{array}{ll}
%                  0 \mbox{\hspace{5mm} if } y^* \leq 2.2 \\
%                 	1 \mbox{\hspace{5mm} if } 2.2<y^* \leq 2.7\\
%                 2 \mbox{\hspace{5mm} if } 2.7<y^* \leq 3.2\\
%                 3 \mbox{\hspace{5mm} if } 3.2<y^* \leq 3.7\\
%                 4 \mbox{\hspace{5mm} if } 3.7 <y^* \leq 4.2\\
%                 5 \mbox{\hspace{5mm} if } 4.2 <y^* \leq 4.7\\
%                 6 \mbox{\hspace{5mm} if } 4.7 <y^* \leq 5.0\\
%                \end{array}
%              \right.
% \end{equation}
%where $y^*$ is the exact Yelp star rating. The model used for this ordered probit is 
%\begin{equation}
%Pr(stars_j = s) = Pr(\kappa_{s-1} < \beta_1 \Delta ln(mw_{j}) + \beta_2 \Delta ln(p_j) + \beta_3 X_j+ \epsilon_j \leq \kappa_s),
%\end{equation}
%where $X_j$ is a vector of covariates including chain status, LS status, employees and sales volume. Using the estimated cutpoints from this model, $\kappa_1, \kappa_2,...,\kappa_s$, the post estimation probabilities of a restaurant having a given star value can be calculated. Table 8 reports these probabilities given a 0 and 10\% minimum wage increase, with LS and chain set to zero and employees and sales volume set to the sample mean. The linear model, equation 2, estimated a significant positive relationship between minimum wage and star rating. The ordered probit results support this estimate, showing that restaurants are less likely to have lower ratings and more likely to have higher ratings after an increase in minimum wage. The largest movements, although still relatively small, occurred within the higher rated restaurants. 



\section{Border Effects}
\subsection{Model}
 To examine the existence of border effects, I first look at the subset of restaurants located within New York City (NYC) and restaurants within the contiguous NJ counties that are located within twelve minutes of the NYC-NJ border.\footnote{The results are persistent further out from the border, but smaller in magnitude.} The specification used to test the existence of border effects is  
\begin{equation}
\begin{split}
\Delta \ln (p_{j,Oct-Apr})  = & \alpha_0 + \alpha_1  \mathbbm{1}(NY=1)  \\
& + \alpha_2 D_{j} + \alpha_3[D_j * \mathbbm{1}(NY=1)]  + \epsilon_{j}, 
\end{split}
\end{equation}
where  $\mathbbm{1}(NY=1)$ is an indicator function denoting if restaurant $j$ is located in NYC, and $D_j$ is the geocoded distance to the closest restaurant on the opposite side of the border. Due to the Hudson River which separates NYC and NJ, the shortest distance between two restaurants on opposite sides of the border is 8 minutes. I therefore normalize the distance for the NYC restaurants to zero as to not count this gap distance twice. I assume that there are no other unobserved variables that are related to both the distance to a border and the change in price. 

For a restaurant in New York, the equation becomes 
$$ \Delta \ln (p_{j,Oct-Apr})  = (\alpha_0 +\alpha_1) +  (\alpha_2 + \alpha_3) D_j  + \epsilon_{j}.   $$
The coefficient $\alpha_2 + \alpha_3$ describes the relationship between distance to the border and price increase for restaurants in NYC close to the New Jersey border. For a restaurant in NJ, the equation becomes 
$$
\Delta \ln (p_{j,Oct-Apr})  = (\alpha_0 ) +  (\alpha_2 ) D_j  + \epsilon_{j}.
$$
The coefficient $\alpha_2$ describes the relationship between distance to the border and price increase for restaurants in NJ close to the NYC border. Figure 3 displays a binned scatter plot of the relationship between distance to the border and price increase. 

\subsection{Estimates}

Table 8 reports the formal statistical results of the border effect models. Columns 1-3 include restaurants in the Yelp dataset. Column 1 shows the estimated relationship between price changes from October `16 to April `17 and the distance to the border for restaurants on the NYC-NJ border. The estimates show that a restaurant 10 minutes further from the border increases prices by 0.27 percentage points more than a restaurant on the border. No significant border effects are reported for NJ restaurants. Column 2 reports this same relationship but for a price change from July `16 to October `16. This estimate is not significant, suggesting that these border effects are not a baseline characteristic of these restaurants. Column 3 explores the existence of within state border effects of restaurants in NYC to the surrounding counties, or NYC MSA. This estimate is much smaller and not significant. Columns 4 and 5 report estimates of border effects using the Grubhub dataset. Column 4 reports that restaurants in NYC 10 minutes further from the border increase prices by 0.19 percentage points more than a restaurant on the NYC-NJ border. The final column reports the within state border effects between NYC and the NYC MSA, which are once again much smaller and not precisely estimated. 


%Three other non-price outcomes are available to analyze in the dataset. Restaurants may change the total number of items that are offered on a menu due to increased labor costs. Since both the Yelp and Grubhub dataset are balanced, using only items that appear at each point in time, the previous item level analysis does not account for any items that have changed over time. Column 1 of Table 9 reports the results of regressing change in total number of items on a menu on log change in minimum wage. The estimate is slightly positive but imprecisely measured. The small point estimate, however, is indicative that restaurants are not changing the total number of items offered due to a minimum wage increase. Another way that restaurants could absorb increased labor costs is to decrease the time the hours of operation. Columns 2 and 3 report results of regressing weekly hours open and number of days open, respectively, on log change in minimum wage. These results show that restaurants decrease the number of hours open per week by approximately 5 hours due to a 10\% increase in minimum wage, and decrease the number of days open per week by 0.2 days. However, neither of these are precisely estimated.

\section{Conclusion}
In this paper, I estimate the effects of a minimum wage increase on restaurants' price and quality. I take advantage of a series of simultaneous minimum wage increases and the growing online presence of restaurants to investigate heterogeneity in price pass through across restaurant characteristics and item type, changes in customer perceived quality, and the existence of border effects. I find that prices increase between 0.3-1.2\% in response to a 10\% increase in minimum wage, results that are consistent with previous literature. Since the data used in this study are primarily non-chain and full service restaurants, these results are not being driven by large franchises and limited service establishments. This is an important conclusion for policy makers, since small businesses are consistently at the forefront of concern. The most conservative pass through estimate, 0.33\%, was calculated using the Yelp dataset. Only a quarter of these restaurants updated their menus throughout the course of the dataset, and thus may be interpreted as a lower bound of the true effects. The pass through results using the subset of restaurants who did update the menus is estimated at 1.22\%, approximately four times that of the full sample. The pass through estimates using the Grubhub dataset, where the menus are necessarily up to date, gave a price pass through estimate of 0.91\%. 



The price pass through estimates differ across restaurant characteristics. For example, restaurants in the lowest quartile of employees reported an estimated 1.0\% pass through while restaurants in the highest quartile of employees reported an estimated 0.32\% pass through. This suggests that smaller establishments have greater autonomy on menu pricing. The magnitude of the price pass through also varies at the item level, where items in categories such as ``Popular'' and ``Drinks'' show a pass through level of over 0.6\%, but items in categories such as ``Pizza'' and ``Dessert'' show significantly lower levels of pass through, with estimates below 0.4\%. This is an important indication for studies in the literature that only examine a small number of items on a menu over time. 

 I find a bimodal effect of an increase in the minimum wage on restaurant quality. Restaurants that were rated at the mean (3.5 stars) or below  prior to the minimum wage increase saw a significant decrease in the star rating given to them by consumers after an increase in minimum wage, where restaurants that started at ratings above the mean saw a positive effect on their consumer ratings due to the increase in minimum wage. The Yelp star rating of a restaurant has been shown to significantly impact a restaurant's profit, thus these estimates are economically significant. Although there is no definitive answer as to what mechanism(s) is (are) driving these effects, I propose some possibilities. The positive relationship of highly rated restaurants could be an indicator that an increase in minimum wage acts as an efficiency wage for higher quality restaurants. The negative relationship, however, could indicate that lower quality restaurants may be cutting corners on food quality, service quality, or employment. Since these ratings are all conditional on consumer expectations, it may also be that an increase in price for lower rated restaurants changes the customer's expectation of what the quality level should be. If this were the case, then true quality may not change for the lower rated restaurants but the quality conditional on the price has changed. In turn the relative quality of higher rated restaurants would be better in comparison to the low rated restaurants. 

Lastly, I examine the extent to which a restaurant's proximity to a minimum wage policy border affects the level of price pass through. Border effects are an area of concern for policy makers when it comes to enacting local minimum wage laws. For restaurants within twelve minutes of a bordering area that is facing a lower minimum wage hike, I find a significant relationship between the restaurant's proximity to the border and the level of price pass through. Specifically, I estimate that a restaurant ten minutes further away from the border increased prices by 0.2 percentage points more than a restaurant on the border. The average price increase over that time period for NYC restaurants close to the border was 0.8\%. Thus this estimate is quite large, and has significant economic implications. Overall, these data have provided the ability to look further into the effects of a minimum wage increase on restaurants and brought about new areas of exploration for further work to be done.






\newpage

\bibliographystyle{apacite}
%\bibliographystyle{unsrt}
\bibliography{cites}

\newpage


\section{Figures}

\begin{figure}[H]
\centering
\includegraphics[scale=1]{map_yelp.pdf}
\caption[Short Heading]{
Counties are color coded by average minimum wage increase, with darker colors representing higher increases in minimum wage. Each red data point represents a Yelp restaurant in the dataset and each black data point represents a Grubhub restaurant in the dataset. Restaurants from both data sources are more highly concentrated in the largely populated areas. 
}
\end{figure}

\begin{figure}[H]
\centering
\includegraphics[scale=.75]{stars_lfits.pdf}
\caption[Short Heading]{
The figures show the relationship between the change in star rating from October 2016 to April 2017 and the change in minimum wage. The starting star value is the Yelp consumer rating of the restaurants in April 2016. All restaurants that have a star rating throughout the dataset are included in the plots. The minimum wage increase reported on the y-axis is the minimum wage change that restaurants faced in January 2017.
}
\end{figure}

\begin{figure}[H]
\centering
\includegraphics[scale=.75]{gh_dist.pdf}
\caption[Short Heading]{
The figure shows the relationship between the change in price from October 2016 to April 2017 and the distance to the NYC- NJ border. Restaurants are binned into 80 quantiles. There is a gap between the NYC and NJ restaurants due to the Hudson river which separates the two states. Distance is measured in driving minutes to the nearest restaurant on the opposite side of the border. 
}

\end{figure}
%
%\begin{figure}
%\centering
%\includegraphics[scale=.75]{star_dens_ny.pdf}
%\caption[Short Header]{
%The plot on the left depicts the distribution of exact Yelp star ratings for large, non chain restaurants in the Newark, NJ area including restaurants in the five contiguous counties to NYC (Essex, Bergen, Hudson, Union and Monmouth) in October of 2016 and in January 2017. The plot on the right depicts the same scenario for restaurants in downtown NYC. These two groups are comparable in geographic location and restaurant characteristics, however Newark restaurants observed a 0.71\% increase in minimum wage while NYC restaurants saw a 22\% increase. There is slight movement with the New Jersey restaurants, which is to be expected since over 70\% of restaurants change ratings between observations. However, the movement in the NYC group is distinctly shift in towards the right, an indication that more restaurants are receiving higher reviews after the minimum wage hike. 
%}
%\end{figure}
%
%\vspace{100mm}
%
%\clearpage
%

\newpage 

\section{Tables}


\begin{table}[H]
\centering
\begin{tabularx}{1\textwidth}{ l c *{6}{Y} } \\ \hline 
%\begin{tabular}{ cccccc } \\ \hline 
 & \multicolumn{3}{c}{Regular Minimum Wage} & \multicolumn{3}{c}{Tipped Minimum Wage}\\
 Area & `16  & `17  & $\% \Delta$ & `16 & `17 &  $\% \Delta$  \\ \hline 
%&&& \\
%1 &  NYC \& FF & \$10.50 & \$12.00 & 14.29\%& - & - & - \\
 %2 & NY Upstate \& FF  & \$9.75 & \$10.75 & 10.26\% & - & -& - \\
 NYC \& Lg & \$9.00 & \$11.00 & 22.22\% & \$7.50 & \$7.50 & 0.00\%\\
 NYC \& Sm & \$9.00 & \$10.50 & 16.67 \% & \$7.50 & \$7.50 & 0.00\%\\
 NYC MSA & \$9.00 & \$10.00 & 11.11\% & \$7.50 & \$7.50 & 0.00\%\\
NY Upstate & \$9.00 & \$9.70 & 7.78\% & \$7.50 & \$7.50  & 0.00\% \\
%7 & Connecticut & \$9.60 & \$10.10 & 5.21\% & \$6.07 & \$6.38 & 5.11\% \\
 New Jersey &  \$8.38 & \$8.44 & 0.72\%  & \$2.13 & \$2.3 & 0.00\% \\
 Massachusetts & \$10.00 & \$11.00 & 10.00\% & \$3.00 & \$3.75  & 25.00\% \\ \hline
%10 & Pennsylvania &  \$7.25 & \$7.25 & 0.00\% & \$2.83 & \$2.83 & 0.00\% \\
%11 & Vermont &  \$9.60 & \$10.00 & 4.2\% & \$4.80 & \$5.00 & 4.2\% \\
\end{tabularx}
%\end{tabular}
\caption[Short Heading]{The regular and tipped minimum wage changes from 2016 to 2017 are reported by group. The first two rows show the minimum wage changes for restaurants in NYC. For the main analysis I use the average of the two minimum wage changes, 19.45\%, for all restaurants in NYC. The NYC MSA group consists of restaurants in the three contiguous counties to NYC: Nassau, Suffolk, and Westchester. NY Upstate encapsulates restaurants in all other areas of the state. NJ and MA minimum wage laws are consistent throughout each state. Although there was a change in tipped minimum wage in Massachusetts, the analysis in this paper will focus only on impacts of changes in the regular minimum wage.
}
\end{table}

\begin{table}[H]
\centering
\begin{center}
\begin{tabular}{lccccc}
\hline  & (1) & (2) & (3) & (4) & (5)\\
 & NYC & NYC MSA & MA & NY Upstate & NJ\\
\hline  \textit{Min Wage Increase}  & 0.194 & 0.111 & 0.100 & 0.078 & 0.007\\
  & (0.000) & (0.000) & (0.000) & (0.000) & (0.000)\\
 \textit{Price (Apr16)}  & 9.781 & 10.703 & 9.891 & 9.622 & 9.869\\
  & (7.030) & (6.545) & (5.541) & (8.544) & (7.737)\\
 \textit{Change Price (Oct16-Apr17)}  & 0.008 & 0.007 & 0.005 & 0.001 & 0.004\\
  & (0.070) & (0.088) & (0.053) & (0.019) & (0.073)\\
 \textit{Increase}  & 0.145 & 0.089 & 0.115 & 0.040 & 0.085\\
  & (0.353) & (0.285) & (0.319) & (0.197) & (0.280)\\
 \textit{Decrease}  & 0.041 & 0.025 & 0.028 & 0.015 & 0.032\\
  & (0.199) & (0.157) & (0.164) & (0.123) & (0.177)\\
 \textit{Price Change $\|$ Increase}  & 0.083 & 0.108 & 0.064 & 0.048 & 0.087\\
  & (0.110) & (0.260) & (0.125) & (0.065) & (0.215)\\
 \textit{Price Change $\|$ Decrease}  & -0.100 & -0.089 & -0.097 & -0.048 & -0.095\\
  & (0.210) & (0.157) & (0.100) & (0.063) & (0.118)\\
 \textit{Eat24 Restaurants}  & 0.309 & 0.250 & 0.201 & 0.141 & 0.224\\
  & (0.462) & (0.434) & (0.401) & (0.348) & (0.417)\\
 \textit{Sales (100k)}  & 8.489 & 7.597 & 9.233 & 5.019 & 5.631\\
  & (14.245) & (37.176) & (19.165) & (5.988) & (9.583)\\
 \textit{Employees} & 10.836 & 10.227 & 14.603 & 10.222 & 9.267\\
  & (15.415) & (20.926) & (24.594) & (11.832) & (15.260)\\
 \textit{Limited Service}  & 0.042 & 0.072 & 0.019 & 0.069 & 0.043\\
  & (0.201) & (0.258) & (0.135) & (0.254) & (0.202)\\
 \textit{Franchise}  & 0.001 & 0.006 & 0.016 & 0.000 & 0.010\\
  & (0.029) & (0.077) & (0.126) & (0.000) & (0.097)\\
\hline  $ N $  & 4242 & 595 & 1658 & 519 & 1793\\
\hline\end{tabular}\\
\end{center}

\caption[Short Heading]{
The means and standard deviations of the primary dataset, Yelp, are reported. Each column contains the restaurants that fall into a specific minimum wage group. All data is balanced at the item level across time periods and aggregated at the restaurant level. The third row reports mean change in natural log of the price, which is approximately the percentage change. The rows titled ``Increase'' and ``Decrease'' report the percentage of restaurants that increased or decreased the averaged menu item price between Oct `16 and Apr '17. The conditional price changes are once again calculated from Oct `16 to Apr `17.
}
\end{table}

\begin{table}[H]
\centering
<<<<<<< HEAD
\begin{center}
\begin{tabular}{lccccc}
\hline  & (1) & (2) & (3) & (4) & (5)\\
 & NYC & NYC MSA & MA & NY Upstate & NJ\\
\hline  \textit{Min Wage Increase}  & 0.194 & 0.111 & 0.100 & 0.078 & 0.007\\
  & (0.000) & (0.000) & (0.000) & (0.000) & (0.000)\\
 \textit{Starting Price (Dec16)}  & 9.356 & 9.759 & 9.346 & 8.640 & 8.949\\
  & (5.589) & (3.513) & (4.182) & (2.787) & (3.761)\\
 \textit{Number of Items}  & 107.790 & 133.704 & 117.472 & 105.053 & 126.917\\
  & (89.516) & (87.262) & (73.827) & (74.815) & (87.014)\\
 \textit{Change Price (Dec16-Apr17)}  & 0.013 & 0.008 & 0.009 & 0.013 & 0.006\\
  & (0.044) & (0.027) & (0.036) & (0.036) & (0.034)\\
 \textit{Increase}  & 0.374 & 0.326 & 0.294 & 0.349 & 0.292\\
  & (0.484) & (0.469) & (0.456) & (0.477) & (0.455)\\
 \textit{Decrease}  & 0.066 & 0.060 & 0.053 & 0.047 & 0.073\\
  & (0.249) & (0.238) & (0.224) & (0.213) & (0.260)\\
 \textit{Price Change $\|$ Increase}  & 0.040 & 0.027 & 0.035 & 0.037 & 0.029\\
  & (0.054) & (0.034) & (0.049) & (0.051) & (0.041)\\
 \textit{Price Change $\|$ Decrease}  & -0.025 & -0.019 & -0.030 & -0.011 & -0.026\\
  & (0.076) & (0.054) & (0.068) & (0.019) & (0.075)\\
 \textit{Sales (100k)}  & 8.643 & 3.218 & 4.976 & 4.684 & 2.987\\
  & (42.233) & (4.072) & (7.722) & (6.363) & (3.007)\\
 \textit{Employees} & 9.916 & 5.650 & 7.996 & 9.582 & 5.130\\
  & (18.763) & (7.000) & (12.253) & (13.502) & (5.172)\\
 \textit{Limited Service}  & 0.042 & 0.070 & 0.029 & 0.067 & 0.065\\
  & (0.202) & (0.255) & (0.168) & (0.251) & (0.247)\\
 \textit{Franchise}  & 0.004 & 0.007 & 0.011 & 0.021 & 0.000\\
  & (0.064) & (0.083) & (0.105) & (0.142) & (0.000)\\
\hline  $ N $  & 4172 & 565 & 866 & 358 & 1320\\
\hline\end{tabular}\\
\end{center}
=======
\begin{center}
\begin{tabular}{lccccc}
\hline  & (1) & (2) & (3) & (4) & (5)\\
 & NYC & NYC MSA & MA & NY Upstate & NJ\\
\hline  \textit{Min Wage Increase}  & 0.194 & 0.111 & 0.100 & 0.078 & 0.007\\
  & (0.000) & (0.000) & (0.000) & (0.000) & (0.000)\\
 \textit{Starting Price (Dec16)}  & 9.356 & 9.759 & 9.346 & 8.640 & 8.949\\
  & (5.589) & (3.513) & (4.182) & (2.787) & (3.761)\\
 \textit{Number of Items}  & 107.790 & 133.704 & 117.472 & 105.053 & 126.917\\
  & (89.516) & (87.262) & (73.827) & (74.815) & (87.014)\\
 \textit{Change Price (Dec16-Apr17)}  & 0.013 & 0.008 & 0.009 & 0.013 & 0.006\\
  & (0.044) & (0.027) & (0.036) & (0.036) & (0.034)\\
 \textit{Increase}  & 0.374 & 0.326 & 0.294 & 0.349 & 0.292\\
  & (0.484) & (0.469) & (0.456) & (0.477) & (0.455)\\
 \textit{Decrease}  & 0.066 & 0.060 & 0.053 & 0.047 & 0.073\\
  & (0.249) & (0.238) & (0.224) & (0.213) & (0.260)\\
 \textit{Price Change $\|$ Increase}  & 0.040 & 0.027 & 0.035 & 0.037 & 0.029\\
  & (0.054) & (0.034) & (0.049) & (0.051) & (0.041)\\
 \textit{Price Change $\|$ Decrease}  & -0.025 & -0.019 & -0.030 & -0.011 & -0.026\\
  & (0.076) & (0.054) & (0.068) & (0.019) & (0.075)\\
 \textit{Sales (100k)}  & 8.643 & 3.218 & 4.976 & 4.684 & 2.987\\
  & (42.233) & (4.072) & (7.722) & (6.363) & (3.007)\\
 \textit{Employees} & 9.916 & 5.650 & 7.996 & 9.582 & 5.130\\
  & (18.763) & (7.000) & (12.253) & (13.502) & (5.172)\\
 \textit{Limited Service}  & 0.042 & 0.070 & 0.029 & 0.067 & 0.065\\
  & (0.202) & (0.255) & (0.168) & (0.251) & (0.247)\\
 \textit{Franchise}  & 0.004 & 0.007 & 0.011 & 0.021 & 0.000\\
  & (0.064) & (0.083) & (0.105) & (0.142) & (0.000)\\
\hline  $ N $  & 4172 & 565 & 866 & 358 & 1320\\
\hline\end{tabular}\\
\end{center}
>>>>>>> 9bf80c4d3367c601bceb3268e37dc31cf9116a6c

\caption[Short Heading]{
The means of the secondary dataset, Grubhub, are reported.  Each column contains the restaurants that fall into a specific minimum wage group. All data is balanced at the item level across time periods and aggregated at the restaurant level. The third row reports mean change in natural log of the price, which is approximately the percentage change. The rows titled ``Increase'' and ``Decrease'' report the percentage of restaurants that increased or decreased the averaged menu item price between Dec `16 and Apr '17. The conditional price changes are once again calculated from Dec `16 to Apr `17.
}
\end{table}


\begin{landscape}
\begin{table}
\centering
%\begin{adjustwidth*}{-2cm}{-2cm}
\begin{center}
\begin{tabular}{lcccccccc}
\hline  & \multicolumn{5}{c}{Yelp} & \multicolumn{3}{c}{Grubhub}\\
 & (1) & (2) & (3) & (4) & (5) & (6) & (7) & (8)\\
 & All & Cntrls & Change & Eat24 & Eat24+GH & All & Cntrls & Eat24\\
\hline  $ Jul16-Oct16 $  & 0.084 & 0.023 & 0.378 & 0.313 & -0.236 &  &  & \\
 & (0.040) & (0.067) & (0.178) & (0.212) & (0.418) &  &  & \\
 $ Oct16-Jan17 $  & 0.171*** & 0.185** & 0.668** & 0.326** & -0.235 &  &  & \\
 & (0.034) & (0.064) & (0.198) & (0.114) & (0.334) &  &  & \\
 $ Jan17-Apr17 $  & 0.162*** & 0.148* & 0.556* & 0.272 & 0.625 &  &  & \\
 & (0.034) & (0.059) & (0.202) & (0.183) & (0.339) &  &  & \\
\hline  $ Dec16-Jan17 $  &  &  &  &  &  & 0.260*** & 0.271*** & 0.206***\\
 &  &  &  &  &  & (0.010) & (0.007) & (0.022)\\
 $ Jan17-Feb17 $  &  &  &  &  &  & 0.245*** & 0.293*** & 0.149**\\
 &  &  &  &  &  & (0.042) & (0.028) & (0.044)\\
 $ Feb17-Mar17 $  &  &  &  &  &  & 0.244*** & 0.295*** & 0.209***\\
 &  &  &  &  &  & (0.022) & (0.015) & (0.044)\\
 $ Mar17-Apr17 $  &  &  &  &  &  & 0.162*** & 0.161*** & 0.032\\
 &  &  &  &  &  & (0.013) & (0.023) & (0.087)\\
\hline \textit{Total Pass Through} & 0.333*** & 0.333*** & 1.224***+ & 0.598** & 0.39 & 0.911*** & 1.019***+ & 0.596***\\
  & (0.066) & (0.114) & (0.394) & (0.281) & (0.434) & (0.053) & (0.069) & (0.166)\\
\hline  $ N $  & 8805 & 5257 & 2099 & 2269 & 432 & 3653 & 2760 & 432\\
 $ NxT $  & 35220 & 21028 & 8396 & 9076 & 1728 & 14612 & 11040 & 1728\\
\hline\end{tabular}\\
\begin{tiny}p<0.1; ** $p<0$.05; *** $p<0$.01; + statistically different than column (1) \end{tiny}\\
\end{center}

%\end{adjustwidth*}
\caption[Short Heading]{
The outcome variable for all columns is the log change in price at the restaurant level due to a 10\% increase in minimum wage. All errors are clustered at the minimum wage group level. Each row represents the amount of pass through occurring in the lag, lead(s), and contemporaneous time periods of the minimum wage changes. Columns 1-5 report estimates using the Yelp data. The first column includes the full sample. The second column includes the vector of controls from the Reference USA dataset. The third column includes only restaurants that changed at least one item over the time period of the dataset. Column 4 includes the subsample of restaurants who employ the Eat24 food delivery service.  The last column of Yelp data include only restaurants who employ the Eat24 food delivery service and are included in the Grubhub dataset. Columns 6-8 report price pass through estimates using the Grubhub sample. Column 6 includes all restaurants, column 7 includes a vector of restaurant control variables. The final column shows estimates for Grubhub restaurants that are also in the Eat24 sample. 
}
\end{table}
\end{landscape}



\begin{table}
\centering
<<<<<<< HEAD
\begin{center}
\begin{tabular}{lcccccc}
\hline  & (1) & (2) & (3) & (4) & (5) & (6)\\
 & Low Sales & High Sales & Low Emps & High Emps & Low Stars & High Stars\\
\hline  $ Apr16-Jul16 $  & 0.288 & 0.182 & 0.249 & 0.122 & 0.189 & 0.026\\
 & (0.145) & (0.271) & (0.147) & (0.260) & (0.076) & (0.114)\\
 $ Jul16-Oct16 $  & 0.249 & -0.073 & 0.286 & 0.019 & 0.260 & 0.153\\
 & (0.182) & (0.204) & (0.140) & (0.245) & (0.056) & (0.126)\\
 $ Oct16-Jan17 $  & 0.260 & 0.096 & 0.364 & 0.032 & 0.429 & 0.189\\
 & (0.126) & (0.121) & (0.119) & (0.155) & (0.033) & (0.080)\\
 $ Jan16-Apr17 $  & 0.468 & 0.334 & 0.238 & 0.386 & 0.361 & 0.190\\
 & (0.122) & (0.142) & (0.132) & (0.176) & (0.094) & (0.103)\\
\hline \textit{Total Pass Through} & 0.728 & 0.43 & 0.602$^+$ & 0.418$^+$ & 0.79 & 0.379\\
  & (0.244) & (0.262) & (0.231) & (0.331) & (0.119) & (0.183)\\
\hline  $ N $  & 1556 & 1723 & 2142 & 1894 & 3832 & 6783\\
 $ NxT $  & 6224 & 6892 & 8568 & 7576 & 15328 & 27132\\
\hline\end{tabular}\\
\begin{tiny} + statistically different than comparison group \end{tiny}\\
\end{center}
=======
\begin{center}
\begin{tabular}{lcccccc}
\hline  & (1) & (2) & (3) & (4) & (5) & (6)\\
 & Low Sales & High Sales & Low Emps & High Emps & Low Stars & High Stars\\
\hline  $ Apr16-Jul16 $  & 0.288 & 0.182 & 0.249 & 0.122 & 0.189 & 0.026\\
 & (0.145) & (0.271) & (0.147) & (0.260) & (0.076) & (0.114)\\
 $ Jul16-Oct16 $  & 0.249 & -0.073 & 0.286 & 0.019 & 0.260 & 0.153\\
 & (0.182) & (0.204) & (0.140) & (0.245) & (0.056) & (0.126)\\
 $ Oct16-Jan17 $  & 0.260 & 0.096 & 0.364 & 0.032 & 0.429 & 0.189\\
 & (0.126) & (0.121) & (0.119) & (0.155) & (0.033) & (0.080)\\
 $ Jan16-Apr17 $  & 0.468 & 0.334 & 0.238 & 0.386 & 0.361 & 0.190\\
 & (0.122) & (0.142) & (0.132) & (0.176) & (0.094) & (0.103)\\
\hline \textit{Total Pass Through} & 0.728 & 0.43 & 0.602$^+$ & 0.418$^+$ & 0.79 & 0.379\\
  & (0.244) & (0.262) & (0.231) & (0.331) & (0.119) & (0.183)\\
\hline  $ N $  & 1556 & 1723 & 2142 & 1894 & 3832 & 6783\\
 $ NxT $  & 6224 & 6892 & 8568 & 7576 & 15328 & 27132\\
\hline\end{tabular}\\
\begin{tiny} + statistically different than comparison group \end{tiny}\\
\end{center}
>>>>>>> 9bf80c4d3367c601bceb3268e37dc31cf9116a6c

\caption[Short Heading]{
The outcome variable for all columns is the log change in price at the restaurant level using the Yelp dataset. All errors are clustered at the minimum wage group level. The reported values compare price pass through of restaurants in the lowest and highest quartiles based on sales, employees, and number of stars in April 2016. Sample sizes are smaller for columns (1)-(4) since these restaurant characteristics rely on the RUSA data. 
}
\end{table}

\begin{landscape}
\begin{table}
\centering
\begin{center}
\begin{tabular}{lcccccccc}
\hline  & (1) & (2) & (3) & (4) & (5) & (6) & (7) & (8)\\
 & All & Popular & Side & Sandwich & Soup/Salad & Entrée & Dessert & Drink\\
\hline  $ Dec16-Jan17 $  & 0.156*** & 0.179*** & 0.165*** & 0.210*** & 0.144*** & 0.147*** & 0.126*** & 0.166***\\
 & (0.008) & (0.004) & (0.013) & (0.004) & (0.003) & (0.002) & (0.016) & (0.003)\\
 $ Jan17-Feb17 $  & 0.160*** & 0.231*** & 0.200** & 0.220*** & 0.067** & 0.155*** & 0.114*** & 0.211***\\
 & (0.020) & (0.003) & (0.053) & (0.005) & (0.017) & (0.013) & (0.022) & (0.029)\\
 $ Feb17-Mar17 $  & 0.146*** & 0.143*** & 0.259*** & 0.231*** & 0.124*** & 0.050*** & 0.102** & 0.036\\
 & (0.011) & (0.014) & (0.019) & (0.014) & (0.005) & (0.006) & (0.023) & (0.041)\\
 $ Mar17-Apr17 $  & 0.014 & 0.065** & 0.067** & 0.024 & -0.003 & 0.013 & 0.014 & -0.002\\
 & (0.010) & (0.015) & (0.017) & (0.037) & (0.008) & (0.007) & (0.024) & (0.018)\\
\hline \textit{Total Pass Through} & 0.477*** & 0.617***$^+$ & 0.691***$^+$ & 0.684***$^+$ & 0.331***$^+$ & 0.365***$^+$ & 0.356*** & 0.412***\\
  & (0.034) & (0.027) & (0.064) & (0.026) & (0.021) & (0.01) & (0.081) & (0.041)\\
\hline  $ N $  & 435676 & 23845 & 53227 & 56750 & 23447 & 81176 & 9963 & 31757\\
 $ NxT $  & 1742704 & 95380 & 212908 & 227000 & 93788 & 324704 & 39852 & 127028\\
\hline\end{tabular}\\
\begin{tiny}p<0.1; ** $p<0$.05; *** $p<0$.01; + statistically different than column (1)\end{tiny}\\
\end{center}

\caption[Short Heading]{
The outcome variable for all columns is the log change in price at the item level using the Grubhub dataset. All errors are clustered at the minimum wage group level. The item categories listed above are mutually exclusive but not exhaustive. The additional food categories not listed include; appetizer, kids, pizza, and other. 
}
\end{table}
\end{landscape}



\begin{landscape}
\begin{table}[H]
\centering
\begin{center}
\begin{tabular}{lcccccccccccc}
\hline  & \multicolumn{2}{c}{All} & \multicolumn{2}{c}{ $ 2.5 $ } & \multicolumn{2}{c}{ $ 3.0 $ } & \multicolumn{2}{c}{ $ 3.5 $ } & \multicolumn{2}{c}{ $ 4.0 $ } & \multicolumn{2}{c}{ $ 4.5$ }\\
 & (1) & (2) & (3) & (4) & (5) & (6) & (7) & (8) & (9) & (10) & (11) & (12)\\
\hline  $ Jul16-Oct16 $  & -0.083 & -0.082 & -1.459 & -1.478 & -0.930 & -0.940 & 0.041 & 0.041 & 1.304 & 1.296 & -0.165 & -0.160\\
 & (0.198) & (0.196) & (1.208) & (1.218) & (0.743) & (0.743) & (0.242) & (0.234) & (0.793) & (0.790) & (0.820) & (0.825)\\
 $ Oct16-Jan17 $  & -0.110 & -0.102 & -0.883 & -0.904 & -1.966* & -1.983* & -0.335 & -0.326 & 1.479* & 1.489* & 1.097 & 1.087\\
 & (0.166) & (0.160) & (2.221) & (2.238) & (0.867) & (0.868) & (0.207) & (0.214) & (0.539) & (0.539) & (0.533) & (0.538)\\
 $ Jan17-Apr17 $  & -0.415 & -0.407 & -2.296 & -2.259 & -1.171 & -1.183 & -0.796* & -0.785* & 0.912 & 0.922 & 0.372 & 0.362\\
 & (0.354) & (0.359) & (1.751) & (1.745) & (1.201) & (1.204) & (0.327) & (0.348) & (0.686) & (0.679) & (0.563) & (0.569)\\
 \textit{Change Price}  &  & -0.028 &  & -0.105* &  & 0.037 &  & -0.032 &  & -0.003 &  & -0.033\\
 &  & (0.025) &  & (0.049) &  & (0.024) &  & (0.031) &  & (0.056) &  & (0.032)\\
\hline \textit{Total \% Change Stars} & -0.526 & -0.509 & -3.179 & -3.163 & -3.136* & -3.166* & -1.131*** & -1.111** & 2.391** & 2.411** & 1.468* & 1.45*\\
  & (0.451) & (0.453) & (3.969) & (3.981) & (2.021) & (2.028) & (0.486) & (0.518) & (1.185) & (1.178) & (1.008) & (1.02)\\
\hline  $ N $  & 6817 & 6815 & 625 & 625 & 1080 & 1080 & 1801 & 1800 & 1904 & 1903 & 982 & 982\\
 $ NxT $  & 27268 & 27260 & 2500 & 2500 & 4320 & 4320 & 7204 & 7200 & 7616 & 7612 & 3928 & 3928\\
\hline\end{tabular}\\
\begin{tiny}\ * $p<0$.1; ** $p<0$.05; *** $p<0$.01\end{tiny}\\
\end{center}

\caption[Short Heading]{
The outcome variable for all columns is the log change in Yelp star rating. All errors are clustered at the minimum wage group level. The star ratings are the rounded Yelp star ratings in April 2016. Restaurants below a 2.5 rating and above a 4.5 rating are not analyzed as subsamples given that they are close to the lower and upper bounds, respectively, and so only have one direction to move. 
}
\end{table}
\end{landscape}

\begin{table}[H]
\centering
\begin{center}
\begin{tabular}{lccccc}
\hline Source & \multicolumn{3}{c}{Yelp} & \multicolumn{2}{c}{Grubhub}\\
 & (1) & (2) & (3) & (4) & (5)\\
Comparison Area &   NJ    &   NJ    & NYC MSA &   NJ    & NYC MSA\\
Time Frame & Oct16-Apr17 & Apr16-Oct16 & Oct16-Apr17 & Dec16-Apr17 & Dec16-Apr17\\
\hline  $ \mathbbm{1}(NY) \quad (\alpha_1) $  & 0.0064 & -0.0064 & 0.0027 & -0.0102 & -0.0134\\
  & (0.0152) & (0.0116) & (0.0355) & (0.0124) & (0.0128)\\
 Distance $\quad (\alpha_2) $  & -0.0009 & 0.0000 & -0.0010 & -0.0000 & 0.0003\\
  & (0.0008) & (0.0006) & (0.0009) & (0.0006) & (0.0004)\\
 Distance * $ \mathbbm{1}(NY) \quad (\alpha_3) $  & 0.0020* & 0.0007 & 0.0016 & 0.0019** & 0.0011\\
  & (0.0011) & (0.0009) & (0.0037) & (0.0009) & (0.0012)\\
 Constant $\quad (\alpha_0) $  & -0.0069 & 0.0041 & -0.0034 & 0.0059 & 0.0137**\\
  & (0.0134) & (0.0102) & (0.0134) & (0.0101) & (0.0059)\\
\hline  $ \alpha_2 + \alpha_3 $  & 0.0011* & 0.0007 & 0.0005 & 0.0018*** & 0.0014\\
  & (0.0008) & (0.0006) & (0.0035) & (0.0007) & (0.0011)\\
\hline  $ N $  & 1002 & 1002 & 522 & 694 & 231\\
\hline\end{tabular}\\
\begin{tiny}\ * $p<0$.1; ** $p<0$.05; *** $p<0$.01\end{tiny}\\
\end{center}

\caption[Short Heading]{
The outcome variable is the percentage point change in price due to a 10 minute change in the distance of a restaurant to the border. The first column includes restaurants in the Yelp dataset within twelve minutes of the NYC - NJ border between October 2016 and April 2017. Column 2 includes these same restaurants but using the change in price from July 2016 to October 2016 as the outcome. Column 3 also uses restaurants in the Yelp datatset but includes only restaurants close to the NYC-NYC MSA border. Columns 4 and 5 include restaurants in the Grubhub dataset. Column 4 reports effects on the NYC-NJ border and column 5 reports effects on the NYC-NYC MSA border. 
}
\end{table}

%
%
%\begin{table}
%\centering
%\begin{tabularx}{1\textwidth}{ c c *{7}{Y} } \\ \hline \hline
%& \multicolumn{6}{c}{Rounded Star Rating (S)} \\
%& $<$2.5 & 2.5 & 3.0 & 3.5 & 4.0  & 4.5 & 5.0 \\ \hline \hline
%$Pr(S)|\Delta ln(mw)=0.00$ & 0.0468 & 0.0957  & 0.1869 & 0.2612 & 0.2643 & 0.1298 & 0.0152 \\
%$Pr(S)|\Delta ln(mw)=0.10$ & 0.0462 & 0.0949 & 0.1860 & 0.2610 & 0.2654 & 0.1312 & 0.0154 \\ \hline
%\% Change & -0.85 & -0.84 & -0.48 & 0.10 & 0.42 & 1.00 & 1.32 \\ \hline \hline
%\end{tabularx}
%\caption[Short Heading]{
%This table reports post estimation probabilities from the ordered probit model. The first column shows predicted probabilities of having the given star rating conditional on no increase in minimum wage. The second column shows predicted probabilities of having the given star rating conditional on a 10\% increase in minimum wage. For these calculations, LS and chain are set at zero and sales and employees are set at the sample mean. 
%}
%\end{table}
%
%\begin{table}
%\centering
%{
\def\sym#1{\ifmmode^{#1}\else\(^{#1}\)\fi}
\begin{tabular}{l*{3}{c}}
\hline\hline
                    &\multicolumn{1}{c}{(1)}&\multicolumn{1}{c}{(2)}&\multicolumn{1}{c}{(3)}\\
                    &\multicolumn{1}{c}{Total Items}&\multicolumn{1}{c}{Hours}&\multicolumn{1}{c}{Days}\\
\hline
$\Delta Ln(mw_{kt}) $&      0.0757         &      -0.477         &     -0.0215         \\
                    &     (3.453)         &     (0.646)         &    (0.0212)         \\
[1em]
Chain               &       3.332         &      0.0643         &      0.0106         \\
                    &     (2.193)         &     (0.387)         &    (0.0127)         \\
[1em]
Employees           &     0.00874         &     0.00326         &   -0.000128         \\
                    &    (0.0173)         &   (0.00314)         &  (0.000103)         \\
[1em]
Sales               &    8.08e-08         &   -3.49e-08         &   -5.73e-10         \\
                    &(0.000000196)         &  (3.51e-08)         &  (1.15e-09)         \\
[1em]
LS                  &       0.615         &       0.226         &    0.000880         \\
                    &     (0.953)         &     (0.190)         &   (0.00624)         \\
\hline
Group FE        &       Yes         &       Yes         &       Yes         \\
Observations        &       23511         &       19356         &       19356         \\
\hline\hline
\multicolumn{4}{l}{\footnotesize Standard errors in parentheses}\\
\multicolumn{4}{l}{\footnotesize \sym{*} \(p<0.05\), \sym{**} \(p<0.01\), \sym{***} \(p<0.001\)}\\
\end{tabular}
}

%\caption[Short Heading]{
%The outcome variable for column 1 is the change in total items per restaurant. The outcome variable for column 2 is the change in total hours open per week, and the the outcome variable for column 3 is the change in total number of days open per week. Although imprecisely estimated, the point estimates show that restaurants are not significantly changing the number of menu items offered, but columns 2 and 3 indicate that restaurants may be cutting back on hours of operation.
%}
%\end{table}
%
%%\begin{table}
%%\centering
%%{
\def\sym#1{\ifmmode^{#1}\else\(^{#1}\)\fi}
\begin{tabular}{l*{4}{c}}
\hline\hline
                    &\multicolumn{1}{c}{(1)}&\multicolumn{1}{c}{(2)}&\multicolumn{1}{c}{(3)}&\multicolumn{1}{c}{(4)}\\
                    &\multicolumn{1}{c}{ch\_av\_p}&\multicolumn{1}{c}{ch\_av\_p}&\multicolumn{1}{c}{ch\_av\_p}&\multicolumn{1}{c}{ch\_av\_p}\\
\hline
$\Delta Ln(MW\_{kt}) $&      0.0248         &      0.0387         &      0.0387         &      0.0264\sym{*}  \\
                    &    (0.0258)         &    (0.0277)         &    (0.0277)         &    (0.0107)         \\
[1em]
Chain               &                     &                     &    -0.00266         &                     \\
                    &                     &                     &    (0.0668)         &                     \\
[1em]
Employees           &                     &                     &   0.0000384         &                     \\
                    &                     &                     &  (0.000274)         &                     \\
[1em]
Sales               &                     &                     &   -2.84e-09         &                     \\
                    &                     &                     &  (2.53e-09)         &                     \\
\hline
Observations        &        4173         &        4173         &        4173         &       23502         \\
\hline\hline
\multicolumn{5}{l}{\footnotesize Standard errors in parentheses}\\
\multicolumn{5}{l}{\footnotesize \sym{*} \(p<0.05\), \sym{**} \(p<0.01\), \sym{***} \(p<0.001\)}\\
\end{tabular}
}

%%\caption[Short Heading]{
%%yelp w/ only gh 
%%}
%%\end{table}

\newpage 



\section{Appendix}

\subsection{Webscrapes}

\subsection{Non-Yelp Menus}

\subsection{Distance Matricies}








\end{document}












